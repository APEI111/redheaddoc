%!TEX TS-program = xetex
%!TEX encoding = UTF-8 Unicode

% 字体
% 份号 密级 紧急程度
\font\heiti="SimHei" at 16pt
\font\songti="[SimSun]:color=0000FF" at 9pt
\font\fangsong="FangSong" at 16pt
\font\kaiti="KaiTi" at 16pt

\font\body="Source Han Serif SC" at 10pt
\body
\XeTeXlinebreaklocale "zh"
\XeTeXlinebreakskip = 0pt plus 1pt minus 0.1pt

\kern -1in
\kern 27mm
\vbox to 35mm{
\heiti
\parindent=0pt
000001\par
密级★期限\par
紧急程度\par
\vfill
}
% 红头标题

{
\parindent=0pt
\font\xiaobiaosong="[SimSun]:color=FF0000" at 36pt
\xiaobiaosong
\hfill%
搬砖科技有限公司软件部文件%
\hfill%
}
\vskip 26pt
{
\fangsong
\centerline{搬软发〔2018〕1号}
}
\vskip 4mm
\bgroup\hrule height 1mm\egroup
% 标题:2号宋体,编排于红色分隔线下空两行(行间距28磅)
% 位置,分一行或多行居中排布;回行时,标题排列应当使用
% 梯形或菱形。
\vskip 28pt
\vskip 28pt
\vbox{
\font\songti = "SimSun" at 22pt%
\baselineskip = 28pt
\parskip = 0pt
\songti
\centerline{搬砖科技软件部}%
\centerline{关于吃葡萄不吐葡萄皮的通知}%
\vskip 28pt
}
{%编排于标题下空一行位置,居左顶格,机关名称后标全角冒号
\fangsong
\parindent = 0pt
各部门主管部门,各鼓励组、各零食补给组、各饮料补给组、各专家以及家属:
}

{
\fangsong
\baselineskip = 28pt
\setbox37 = \hbox{缩进}
\parindent = \wd37
关于强化复工企业及员工疫情防控措施的通知\par
根据中央和省、市、区有关复工复产的要求,在疫情防控期间,请各园区和广大企业认真落实联防联控、群防群控职责,配合做好以下重点事项:\par
{\heiti 一、园区防控措施}\par
{\kaiti (一)出入管理}\par
所有进入企业人员必须正确佩戴口罩、必须进行体温检测,并积极配合园区主入口的问询或登记。疫情防控期间,外卖人员一律不得进入园区。\par
{\kaiti (二)体温测量}\par
园区主入口、各楼宇大堂设专人值守测量体温,凡体温超过37.3℃、未正确佩戴口罩、重点区域返园区未居家或集中隔离观察14天及不配合检查的人员禁止进入。同时,建议园区在员工进入办公区域时,由园区进行体温检测。\par
{\kaiti (三)楼宇管理}\par
园区将对公共区域进行每日2次的消毒防疫;在公共部位集中设置“废弃口罩存放箱”,张贴明显标识,每天由保洁员集中处理。\par
{\heiti 二、企业防控措施}\par
{\kaiti (一)企业防控机制}\par
请企业自觉落实疫情防控单位主体责任,建立企业单位法人或主要负责人任组长的疫情防控领导小组,周密制定疫情防控措施和应急预案,建立“一员一档”,并严格落实执行。灵活安排员工分批返岗,在做好疫情防控工作的前提下,有序组织复工复产;建议结合各自实际采取弹性工时、远程办公、居家办公等措施。\par
{\kaiti(2)企业人员管理}\par
对有湖北地区往来史、与湖北地区人员有接触史的员工,请企业企业加强关心沟通联系,建议他们在疫情稳控后回企;对有以上情况并已经抵企的员工,督促其及时联系所在街镇的村/居委会或入住的酒店等,采取居家或集中隔离医学观察不少于14天等防控措施。疫情防控期间,对外地(国)返企同时采取不少于14天的隔离防控措施。\par
{\kaiti(3)防疫物资管理}\par
请企业切实关注员工的身体健康状况,尽可能做好相关防疫物资(口罩、体温检测仪、消毒剂等)配备。请正确使用防疫物资,按要求做好保管工作,确保安全存放及使用。\par
{\kaiti(4)办公场所管理}\par
复工前,请企业对工作场所进行消毒处理,并确保每日消毒。进入企业请佩带口罩,对上班员工做到每日2次体温检测,并做好记录。企业自身有班车运营的请做好员工体温检测、口罩佩戴等基础工作。建议线上会议沟通工作,减少集中开会。如需开会,必须佩戴口罩,入会议室前洗手消毒,控制时长,室内通风排浊。\par
{\kaiti(5)企业访客管理}\par
企业访客应佩戴口罩,进入办公区前先检测体温,确定有无重点疫区接触史和发热、咳嗽、呼吸不畅等症状。无上述情况且体温正常,方可进入办公区并在指定区域会客。疫情防控期间,建议企业尽量减少安排员工外出以及出差。\par
{\kaiti(6)防疫知识宣传}\par
请企业及时向员工普及新冠肺炎相关知识,广泛宣传落实预防措施,教育员工不传谣、不信谣,避免员工出现恐慌等不良情绪。建议企业向本企业的湖北籍员工发一条关爱信息、打一个暖心电话,做到隔离不隔心。\par
{\heiti 三、个人防护温馨提示}\par
为避免复工后入园测温工作可能导致的车辆、人员拥堵,建议各企业根据实际情况尽量选择错峰上班。\par
为避免乘坐电梯时可能造成的感染风险,建议低楼层的企业提倡鼓励员工使用楼梯步行上下楼。\par
为避免外出就餐和使用外卖可能造成的感染风险,建议企业提倡员工自行带饭、预约用餐或错峰用餐。\par
凡来自或者途经疫情重点地区的人员进入本企业的,以及与上述人员、确诊或疑似新冠肺炎病例有密切接触的人员,应主动接受体温检测,自觉实施居家或者积极配合集中隔离医学观察14天。对未按照规定主动登记,在工作人员询问时不如实告知,或者拒绝执行相关检测、居家隔离、集中隔离观察措施的,将按照有关规定处理,构成治安管理违法行为或者犯罪的,公安机关将依法追究相关人员的法律责任。\par
请各经济开发园区和广大企业将疫情防控和复工复产情况每日5点前用办公网公务邮箱报送至区疫情防控指挥部办公室。\par

\vskip 2em
\setbox37=\hbox{右空四字}
{\songti\hfill TODO:盖章,居中对齐\kern \wd37\par}
\hfill 某某区疫情防控指挥部办公室\kern \wd37\par
\hfill 2020年8月19日\kern \wd37\par
}
\bye
