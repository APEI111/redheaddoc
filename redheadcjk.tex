\documentclass{article}
\usepackage[vmargin=2cm,hmargin=3cm]{geometry}

\usepackage[CJKspace]{xeCJK}
\usepackage{xpinyin}

\setmainfont{SimSun}
\setsansfont{SimSun}
\setmonofont[Scale=MatchLowercase]{SimSun}

\setCJKmainfont{SimSun}
\setCJKsansfont{SimSun}
\setCJKmonofont{SimSun}


%% From https://tex.stackexchange.com/a/379867/226
\def\biangT{{\xeCJKsetup{CJKglue={}}⿺辶⿳穴⿰月⿰⿲⿱幺長⿱言馬⿱幺長刂心}}
\def\biangS{{\xeCJKsetup{CJKglue={}}⿺辶⿳穴⿰月⿰⿲⿱幺长⿱言马⿱幺长刂心}}

\usepackage{multicol}
\usepackage[colorlinks]{hyperref}

\begin{document}

The Noto Serif/Sans CJK fonts (also known as Source Hans Serif 思源宋体、Source Hans Sans思源黑体)have a  huge glyph coverage of ideographs. Here's a sampler using a short excerpt from the lyrics of 《生僻字》(\emph{`Rare words'}) written by 陈柯宇 Chen Keyu, which contains a large number of rare and archaic Chinese ideographs.

\begin{multicols}{2}
\begin{verse}
茕茕孑立,沆瀣一气。\\
踽踽独行,醍醐灌顶。\\
绵绵瓜瓞,奉为圭臬。\\
龙行龘龘,犄角旮旯。\\
娉婷袅娜,涕泗滂沱。\\
呶呶不休,不稂不莠。
\end{verse}

\sffamily
\begin{verse}
咄嗟、蹀躞、耄耋、饕餮;\\
囹圄、蘡薁、觊觎、龃龉;\\
狖轭鼯轩,怙恶不悛。\\
其靁虺虺,腌臢孑孓。\\
陟罚臧否,针砭时弊。\\
鳞次栉比,一张一翕。
\end{verse}
\end{multicols}

Test for Traditional Chinese: (Compare the punctuations to that of Simplified Chinese.)

\begin{multicols}{2}
% \CJKfontspec{Noto Serif CJK TC}
\begin{verse}
兩個黃鸝鳴翠柳,\\一行白鷺上青天。\\
窗含西嶺千秋雪,\\門泊東吳萬里船。
\end{verse}

% \CJKfontspec{Noto Sans CJK TC}
\begin{verse}
兩個黃鸝鳴翠柳,\\一行白鷺上青天。\\
窗含西嶺千秋雪,\\門泊東吳萬里船。
\end{verse}
\end{multicols}

Noto Serif and Sans CJK also have glyphs for \href{https://blogs.adobe.com/CCJKType/2017/04/designing-implementing-biang.html}{\pinyin{biang2}}, used in the name of a \href{https://en.wikipedia.org/wiki/Biangbiang_noodles}{Chinese noodle dish}. This ideograph has yet to be encoded in UTF so it's not really supported by any input methods! Instead, we're using the GSUB trick mentioned in this \href{https://blogs.adobe.com/CCJKType/2017/04/designing-implementing-biang.html}{Adobe blog post} and this \href{https://tex.stackexchange.com/a/379867}{\LaTeX{} code snippet}, to render it here:
\begin{center}\Huge
\biangT\biangS \quad \sffamily\biangT\biangS
\end{center}

Test for Japanese:
\begin{multicols}{2}
% \CJKfontspec{Noto Serif CJK JP}
\begin{verse}
秋来ぬと、\\
目にはさやかに見えねども、\\
風の音にぞおどろかれぬる。
\end{verse}

% \CJKfontspec{Noto Sans CJK JP}
\begin{verse}
秋来ぬと、\\
目にはさやかに見えねども、\\
風の音にぞおどろかれぬる。
\end{verse}
\end{multicols}


Test for Korean:
\begin{multicols}{2}
% \CJKfontspec{Noto Serif CJK KR}
\begin{verse}
오늘이 오늘이소서 매일이 오늘이소서\\
저물지도 새지도 말으시고\\
새려면 늘 언제나 오늘이소서
\end{verse}

% \CJKfontspec{Noto Sans CJK KR}
\begin{verse}
오늘이 오늘이소서 매일이 오늘이소서\\
저물지도 새지도 말으시고\\
새려면 늘 언제나 오늘이소서
\end{verse}
\end{multicols}

The Noto CJK fonts respects regional writing conventions for the same ideograph. Here are the same ideograph 述typeset with Noto Serif CJK for different regions:

\begin{center}
这里的述字有四种写法。
% [SC] {\CJKfontspec{Noto Serif CJK SC}述}\quad
% [TC] {\CJKfontspec{Noto Serif CJK TC}述}\quad
% [JP] {\CJKfontspec{Noto Serif CJK JP}述}\quad
% [KR] {\CJKfontspec{Noto Serif CJK KR}述}
\end{center}

You can set a default CJK serif and sans font for your document by loading the \texttt{xeCJK} package, then use the \verb|\setCJKmainfont{...}| and \verb|\setCJKsansfont{...}| commands. (See preamble of this example project.) You can always switch to a different CJK font in the middle of your document using \verb|\CJKfontspec{...}|. Remember to compile your project \href{https://www.overleaf.com/learn/latex/Choosing_a_LaTeX_Compiler}{with the XeLaTeX or LuaLaTeX compilers}.

\end{document}
