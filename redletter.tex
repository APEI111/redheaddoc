%!TEX TS-program = xetex
%!TEX encoding = UTF-8 Unicode

\font\1="Source Han Serif SC:color=0000FF"/OT at 10.5bp
\font\2="Noto Sans SC:color=0000FF"/OT at 10.5bp
\font\err="Noto Sans SC:color=FF0000" at 10.5bp
\def\hr{\special{color push rgb 0 0 1}\hrule\special{color pop}}

\1

\XeTeXlinebreaklocale "zh"
\XeTeXlinebreakskip = 0pt plus 1pt minus 0.1pt

% 方正字体
\font\songti="Source Han Serif SC"/OT at 16bp
\font\fangsong="Source Han Serif SC"/OT at 16bp
\font\heiti="Source Han Serif SC"/OT at 16bp
\font\title="方正小标宋简体:color=FF0000"/OT at 32bp

% 思源宋体
\font\songti="Source Han Serif SC"/OT at 16bp
\font\fangsong="Source Han Serif SC"/OT at 16bp
\font\heiti="Source Han Serif SC"/OT at 16bp
\font\title="Source Han Serif SC Bold:color=FF0000"/OT at 32bp

% 中易字体
\font\songti="SimSun" at 16bp
\font\fangsong="FangSong" at 16bp
\font\heiti="SimHei" at 16bp
\font\title="Source Han Serif SC Bold:color=FF0000"/OT at 32bp

% GB/T9704—2012

%\hr
% 纸张设置
% 公文用纸采用GB/T148中规定的A4型纸,其成品幅面尺寸为:210mm×297mm.
\pdfpagewidth 210mm
\pdfpageheight 297mm

% 版心设置
\hsize=158mm
\hoffset=3pt
\vsize=225mm
\voffset=27pt

\footline{\1 GB/T9704-2012\hfill}

{% 发文机关
\title
\centerline{中华人民共和国XXXXX部}\par
\vskip 4mm
}

{% 双红线
\special{color push rgb 1 0 0}
\hrule height 3pt
\vskip 2pt
\hrule height 1pt
\special{color pop}
}
%%%%%%%
\vbox to 14pt{
% \1 第一个要素与该线的距离为3号汉字高度的7/8。
% \vfill
% \hr
}
%%%%%%%
\hbox{
\parindent=0pt
\baselineskip=28pt

\vbox to 72pt{% 份号、密级和保密期限
\heiti
\setbox37=\hbox{123456}
\hsize=\wd37
000001\par
机\hfill 密\par
特\hfill 急\par
}
\hfill
\vbox to 72pt{% 发文字号
\fangsong
\hsize=136mm% TODO: 自动右对齐版芯
\rightline{XXX〔2012〕10号}
\strut\par
\strut\par
}
}
{% 标题
\heiti
\vbox to 56.21pt{
%\hr
%\vfill
%\1 标题居中编排,与其上最后一个要素相距二行。
%\vfill
%\hr
}
\centerline{XXXXX关于XXXXXXX的通知}
}
{% 正文
\fangsong
\baselineskip=28.105pt
\parindent=32.12pt
% {\tt \the\hsize * \the\vsize}
\noindent XXXXXXXX:\par
\noindent{}一二三四五六七八九十一二三四五六七八九廿一二三四五六七八九卅(每行28字,共22行)。\par
\noindent 10.1信函格式\par
发文机关标志使用发文机关全称或者规范化简称,居中排布,上边缘至上页边为30mm,推荐使用红色小标宋体字。联合行文时,使用主办机关标志。\par
发文机关标志下4mm处印一条红色双线(上粗下细),距下页边20mm处印一条红色双线(上细下粗),线长均为170mm,居中排布。\par
如需标注份号、密级和保密期限、紧急程度,应当顶格居版心左边缘编排在第一条红色双线下,按照份号、密级和保密期限、紧急程度的顺序自上而下分行排列,第一个要素与该线的距离为3号汉字高度的7/8。\par
发文字号顶格居版心右边缘编排在第一条红色双线下,与该线的距离为3号汉字高度的7/8。\par
标题居中编排,与其上最后一个要素相距二行。\par
第二条红色双线上一行如有文字,与该线的距离为3号汉字高度的7/8。\par
首页不显示页码。\par
版记不加印发机关和印发日期、分隔线,位于公文最后一面版心内最下方。\par
}
%%%%%%%%%%%%%%%%%%%%%%%%%%%%%%%%%%%%%%%%%%%%%%%%%%%%%%%%%%%%%%%%%%%%%%%%%%%%%%%%
\vfill

\centerline{\2 说\quad 明}
\setbox37=\hbox{缩进}
\parindent=\wd37
\medskip

\noindent 换算行间隔({\tt\char92 baselineskip}):\par
$$ 16.06pt \times 1.75 = 28.105pt(10.56748mm)$$
\noindent 另外,版心高225毫米,共22行,计算行间距
$$	21(16.06pt+x) + 16.06pt = 225mm, 1pt=0.376mm $$
\noindent 解得:
$$ 16.06pt + x = 28.4954pt,225mm = 598.4043pt $$
\par
\noindent 重新代入
$$ 21 \times 28.5pt + 16.06pt = 614.56pt $$
\noindent 代码里面按$28pt$的行间距:
$$ 28pt \times 21 + 16.06pt = 604.06pt$$
请在$28.1pt$\char126 $28.4954pt$之间适当调节,使得版芯刚好$22$行文字。\par

\noindent 本文档对国家标准{\tt GB/T9704-2012}实现程度:\par
天头:37mm±1mm, $72pt + \the\voffset \approx 37.224mm $\par
订口:28mm±1mm, $72pt + \the\hoffset \approx 28.200mm $\par
版心:{$156mm\times 225mm$}, $(\the\hsize\times\the\vsize )\times 0.376mm/pt \approx ?$\par
字体:三号,\kern -5pt
% 实施时用的字体
{\songti 宋体}\setbox37=\hbox{\songti 字}\the\wd37、\kern -5pt
{\heiti 黑体}\setbox37=\hbox{\heiti 字}\the\wd37、\kern -5pt
{\fangsong 仿体}\setbox37=\hbox{\songti 字}\the\wd37、\kern -5pt
{\title 小标宋}\setbox37=\hbox{\title 字}\the\wd37。\par
距离:3号汉字7/8,$16.06pt \times {7 \over 8} = 14.0525pt \approx 14pt$\par

\medskip
\noindent 任务:\par
底部双红线位置,\dots\par%
疑问:
是否每一页的底部需要添加双红线?%


\hr
\medskip
{\noindent TODO:\par
水平盒子的分散对齐\par
竖直盒子的摆放\par



\medskip
{
\noindent 纠错与讨论:\par
\TeX{}讨论组:\tt 273{\kern 1pt}409{\kern 1pt}844\par
\1 北理工\TeX{}用户组:\tt 1703{\kern 1pt}68921\par
\1 电话:\tt(+86)138{\kern 1pt}232{\kern 1pt}25852\par
}

}
%%%%%%%%%%%%%%%%%%%%%%%%%%%%%%%%%%%%%%%%%%%%%%%%%%%%%%%%%%%%%%%%%%%%%%%%%%%%%%%%
{% 双红线
\vbox to 14pt{
%\hr
%第二条红色双线上一行如有文字,与该线的距离为3号汉字高度的7/8。
}
\special{color push rgb 1 0 0}
\hrule height 1pt
\vskip 2pt
\hrule height 3pt
\special{color pop}
}
\vskip 5cm
\par
\centerline{\2 使用说明}
\noindent{\tt Windows}\par
北大方正:
把字体文件放置到如下目录,如果没有该目录,请新建一个。\par
{\tt C:{\char92}texlive{\char92}2020{\char92}texmf-dist{\char92}fonts{\char92}opentype{\char92}founder}\par
用\TeX{}Live打开编译。\par
\noindent{\tt Linux}\par
{\tt	xetex redletter.tex}

\noindent{\tt MacOS}\par

\centerline{\2 版权说明}
方正字体,思源字体,中易字体。


\bye
