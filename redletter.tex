%!TEX TS-program = xetex
%!TEX encoding = UTF-8 Unicode

\font\1="Source Han Serif SC:color=0000FF"/OT at 10.5bp
\font\err="Noto Sans SC:color=0000FF" at 10.5bp
\def\hr{\special{color push rgb 0 0 1}\hrule\special{color pop}}

\1

\XeTeXlinebreaklocale "zh"
\XeTeXlinebreakskip = 0pt plus 1pt minus 0.1pt

\hr
{% 发文机关
中华人民共和国XXXXX部\par
}
\hrule height 3pt
\vskip 2pt
\hrule height 1pt
{% 份号、密级和保密期限
000001\par
机密\par
特急\par
}
{% 发文字号
XXX〔2012〕10号\par
}
\hr

{% 标题
XXXXX关于XXXXXXX的通知
}
\hr
XXXXXXXX:\par
XXXXXXXXXXXXXXXXXXXXXXX。\par
XXXXXXXXXXXXXXXXXXXXXX。\par
XXXXXXXXXXXXXXXXXXXXX。\par
XXXXXXXXXXXXXXXXXXXX。\par
一二三四五六七八九十一二三四五六七八九廿一二三四五六七八九卅。\par

10.1信函格式\par
发文机关标志使用发文机关全称或者规范化简称,居中排布,上边缘至上页边为30mm,推荐使用红色小标宋体字。联合行文时,使用主办机关标志。\par
发文机关标志下4mm处印一条红色双线(上粗下细),距下页边20mm处印一条红色双线(上细下粗),线长均为170mm,居中排布。\par
如需标注份号、密级和保密期限、紧急程度,应当顶格居版心左边缘编排在第一条红色双线下,按照份号、密级和保密期限、紧急程度的顺序自上而下分行排列,第一个要素与该线的距离为3号汉字高度的7/8。\par
发文字号顶格居版心右边缘编排在第一条红色双线下,与该线的距离为3号汉字高度的7/8。\par
标题居中编排,与其上最后一个要素相距二行。\par
第二条红色双线上一行如有文字,与该线的距离为3号汉字高度的7/8。\par
首页不显示页码。\par
版记不加印发机关和印发日期、分隔线,位于公文最后一面版心内最下方。\par

\hrule height 1pt
\vskip 2pt
\hrule height 3pt

\bye
